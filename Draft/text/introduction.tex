\section{Introduction \todo{all}} 
With the second run of LHC in progress, the main goal the community have is to
better measure, and describe, the features characterising the so much Standard-Model
look alike Higgs boson. In order to achieve this goal, we need to increase our power
to discriminate between what we define as the {\it signal} and what then is the {\it background}, the former being the part of sample which has the features we would like to measure while the latter is usually something we would like to get rid of.
Amongst the various production mechanisms, the Vector-Boson-Scattering(-Fusion) or simply VBF, provides not only one of the largest production rates but also a very clean signature from the QCD background.

There are various phenomenological reasons to further increase the sensitivity for such processes. Firstly, even if the total rate is about the 8\% of the gluon-fusion cross section, it is expected
that the total cross section for VBF will increase by a factor of $\sim$ 3.3 (with its backgrounds also growing) with the increase of energy. Secondly, diagrams that involve a high mass Higgs in the VBF processes give access to the longitudinal modes of the produced vector bosons, and as these modes are only present in a Electro-Weak symmetry breaking scenario, these process provide an ideal framework to precisely study the intrinsic quantities of such mechanism.

